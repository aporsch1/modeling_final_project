\documentclass[12pt, oneside]{article}

\usepackage[utf8]{inputenc}
\usepackage[T1]{fontenc}
\usepackage{textcomp}
\usepackage{amsmath, amssymb}
\usepackage{fullpage}
\usepackage{graphicx}
\usepackage{amssymb, amsmath, amsthm}
\newtheorem{theorem}{Theorem}[section]
\newtheorem{corollary}{Corollary}[theorem]
\newtheorem{lemma}{Lemma}[theorem]
\title{Modeling Computer Viruses in Peer-to-Peer Networks}
\author{Abraham Porschet}
\date{\today}


% figure support
\usepackage{import}
\usepackage{xifthen}
\pdfminorversion=7
\usepackage{pdfpages}
\usepackage{transparent}

\pdfsuppresswarningpagegroup=1

\begin{document}
    \maketitle

    \section{Introduction}
        Peer to peer networks, caught on in the late 1990s and early 2000s with platforms like Limewire, Napster and MySpace. These websites
        were illustrative of the fun people could have with the internet, and also the malicious actions people could take using the internet.
        Peer to peer networks can be susceptible to various types of viruses, including self-propogating viruses such as the Samy worm \cite{VICE_2015}.\newline
        
        Obviously the websites listed above have differences, but the main connection between them is that there are multiple users across a network
        who each have their own files or links or images publicly available for others to view or download. If a user wants to visit another user's 
        page or download another user's file(s), the user's web-client will have to connect with the other user and will then do what it was asked to do.\newline

        Viruses have had a lot of success against networks like these, people can download things without being aware of it, and then viruses can download a payload to their 
        computer that adds itself to her files (so other people can download it by accident) and alters her system somehow. These viruses can behave quite similarly to 
        *actual* viruses. This means that a whole set of models, epidemiological models, are on the table to be used.\newline
        
        P2P (peer-to-peer) networks, used commercially, have millions of users that are connected in a web of millions of computer systems.
        In this web, file transfers happen rapidly, often in times much less than a second. When these webs encounter worms like the Samy worm
        or other similar viruses, they can propagate to thousands of users in very little time. If people are able to predict the spread of
        a virus inside of P2P networks, they are able to better respond to the virus at hand and better understand how to secure their network more.
        This modeling will take place in a simulation of P2P networks to assess a network's potential for propagation of viruses.\newline

    \section{Methods and Model}
    
        The intent of the model is to approximate the spread of a computer virus that exists in a Peer-to-Peer (P2P) network. Note that we use the term user
        to refer to a person using a P2P client program. The term peer is used to collectively refer to a P2P client and the user directing its behaviour.
        The model examined \cite{1689197}, from Thommes and Coates, has peers of three types, Susceptible (S), Exposed (E), and Infected (I). 
    \subsection{Model}
        Since there are three different categories that a peer can fall into and each category is mutually exclusive with the others,
        it follows that at any time $t$, that  $N=S(t)+E(t)+I(t)$ where  $N$ is the number of peers in the system. We also make the assumption that the number of uninfected files is fixed at some $M$.
        The number of infected files at time  $t$ is given by a function  $K(t)$. When we then want to calculate the proportion of infected
        files to uninfected files, we get the proportion  $q(t)=\frac{K(t)}{M+K(t)}$. We then assume that when a user interacts with a file,
        that it has a probability dependent on the number of infected files on the network. Another assumption made with this model
        is that the probability of an arbitrary file being infected is time invariant, and is only dependent on the proportion $q(t)$.
        We call the function that determines this probability  $f{q(t)}$. Then, for the last introduction of new variables, we have three
        other parameters for the following three actions: a peer downloading a file from another peer, a peer executing a shared file, 
        and a peer recovering. The parameters are then,  $\lambda_S$:  the rate in files per minute at which a peer downloads new files, 
        $\lambda_E$: average rate, in files per minute, at which each peer executes files, our last rate parameter is  $\lambda_R$:
        which is the average rate of recoveries per minute, at which infected peers recover. A recovery occurs when all infected files are
        removed, returning the peer to susceptible. The state cycle of this model is  $S\to E\to I \to S\ldots$\newline

        We now derive our differential equations.
        \subsubsection{Rate at Which Infected Peers Change}
        When infected peers change back to susceptible,  $I$ decreases by 1. Recoveries occur at rate  $\lambda_R I(t)$.
        Similarly, when an exposed person executes a file, they become infected and this happens at the rate $\lambda_E E(t)$, which
        gives us our first differential equation \[
        \frac{dI(t)}{dt}=-\lambda_R I(t)+\lambda_E E(t)
        \] 
        \subsubsection{Rate at Which Exposed Peers Change}
        The rate that exposed peers decrease due to infection, is just the negative of the $E(t)$ term from the last differential equation.
        The rate at which susceptible peers are exposed is dependent on the rate at which they download files, multiplied by the probability
        that a given file is infected. This gives us a rate of \[
            \frac{dE(t)}{dt}=-\lambda_E E(t)+\lambda_S S(t)f{q(t)}
        \]
        \subsubsection{Rate at Which Susceptible Peers Change}
        This is determined by the negative of the first term of the first equation and the negative of the second term of the second equation, giving us
        \[
            \frac{dS(t)}{dt}=-\lambda_S S(t)f{q(t)}+\lambda_R I(t)
        \]
    \subsection{results}
    In order to find an equilibrium, if one exists, in this network we have to find a $t=T$ 
    such that $\frac{dE(T)}{dt}=\frac{dI(T)}{dt}=\frac{dS(T)}{dt}=0$. Thommes and Coates denote the equilibrium
    values for $E,I$, and  $S$ as  $\hat{E}, \hat{I}$ and $\hat{S}$


\bibliography{paper}
\bibliographystyle{plain}
\end{document}
